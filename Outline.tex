\documentclass{ctexart}
\usepackage{listings}
\usepackage{fancyhdr}
\usepackage{geometry}
\usepackage{indentfirst}

\pagestyle{fancy}
\fancyhead[]{}
\lfoot{}
\renewcommand\headrulewidth{0pt}

\setlength{\parindent}{2em}
\geometry{a4paper,scale=0.8}
\title{仰海锐老师班马原提纲}
\author{计算机科学与技术系 \ 朱卓然}
\date{\today}
\begin{document}
\maketitle

\section{名词解释}~{}
\subsection{马克思主义}~{}
马克思主义是由马克思和恩格斯创立并为后继者所不断发展的科学理论体系,
是关于自然、社会和人类思维发展一般规律的学说,是关于社会主义必然代替
资本主义、最终实现共产主义的学说,是关于无产阶级解放、全人类解放和每个人
自由而全面发展的学说,是无产阶级政党和社会主义国家的指导思想,是指引人民
创造美好生活的行动指南。

\subsection{交往}~{}
交往是唯物史观的重要范畴,指一定历史条件下的现实的个人、群体、阶级、民族、国家
之间在物质和精神上的相互往来、相互作用、彼此联系的活动。

\subsection{世界历史}~{}
唯物史观视域中的“世界历史”是指民族、国家通过普遍交往,打破孤立隔绝的
状态,进入相互依存、相互联系的世界整体化的历史。在马克思、恩格斯看来
,资本主义生产方式的发展和交往的普遍化推动了历史向世界历史的转变。

\subsection{金融资本}~{}
金融资本是由工业垄断资本和银行垄断资本融合在一起而形成的一种
垄断资本。
\subsection{金融寡头}~{}
金融寡头是指操纵国民经济命脉,并在实际上控制国家政权的少数垄断资本家或
垄断资本家集团。

\subsection{垄断利润}~{}
垄断利润是垄断资本家凭借其在社会生产和流通中的垄断地位而获得的超过平均利润的
高额利润。
\subsection{垄断价格}~{}
垄断价格包括垄断高价和垄断低价两种形式。垄断高价是指垄断组织出售商品时规定的高于生产价格的价格;
垄断低价是指垄断组织在购买非垄断企业所生产的原材料等生产资料时规定的低于生产价格的价格。

\section{简答题}~{}
\subsection{感性认识和理性认识}~{}
\vspace{-5mm}
\subsubsection{感性认识和理性认识的概念}~{}
\vspace{-5mm}
\begin{itemize}
\item[$\bullet$]感性认识是认识的初级阶段,是人们在实践的基础上,通过感官直接感受到的关于事物的现象、事物的外部联系、事物的各个方面的认识。
\item[$\bullet$]理性认识是认识的高级阶段,是人们借助抽象思维,在概括整理大量的感性材料的基础上,达到关于事物的性质、全体、内部联系和事物自身规律性的认识。 
    
\end{itemize}

\subsubsection{感性认识和理性认识的辩证关系}~{}
\vspace{-5mm}
\\
相互区别:
\begin{itemize}
\item[$\bullet$]第一,对象不同。感性认识是对事物外部联系的反映,而理性认识是对事物内部联系的反映。
\item[$\bullet$]第二,形式不同。感性认识包括感觉、知觉和表象三种形式,而理性认识包括概念、判断和推理三种形式。
\item[$\bullet$]第三,特点不同。感性认识的特点是直接性和具体性,而理性认识的特点是间接性和抽象性。    
\end{itemize}
相互联系:
\begin{itemize}
    \item[$\bullet$]第一,理性认识依赖于感性认识。
    \item[$\bullet$]第二,感性认识有待于发展和深化为理性认识。
    \item[$\bullet$]第三,感性认识和理性认识相互渗透、相互包含。
    \end{itemize}

\subsubsection{从感性认识到理性认识的条件}~{}
\vspace{-5mm}
\begin{itemize}
\item[$\bullet$]第一,投身实践,深入调查,获取十分丰富和合乎实际的感性材料。这是实现由感性认识上升到理性认识的基础。
\item[$\bullet$]第二,经过思考的作用,运用理论思维和科学抽象,对丰富的感性材料进行去粗取精、去伪存真、由此及彼、由表及里的加工制作,形成概念和理论的系统。
\end{itemize}

\subsection{价值评价}~{}


\subsubsection{价值评价的特点}~{}
\vspace{-5mm}
\begin{itemize}
\item[$\bullet$]第一,评价以主客体的价值关系为认识对象。
\item[$\bullet$]第二,评价结果与评价主体直接相关。
\item[$\bullet$]第三,评价结果的正确与否依赖于对客体状况和主体需要的认识。
\item[$\bullet$]第四,价值评价有科学与非科学之别。  
\end{itemize}

\subsubsection{价值评价的标准}~{}
对于任何主体而言,是否推动社会历史进步,是否符合社会发展趋势,
是否维护、满足了最广大人民的需要和根本利益,是价值评价的
最高标准,是判断特定主体实际需要是否合理的最高尺度。

\subsection{价值规律}~{}


\subsubsection{价值评价的特点}~{}
\vspace{-5mm}
\begin{itemize}
\item[$\bullet$]第一,评价以主客体的价值关系为认识对象。
\item[$\bullet$]第二,评价结果与评价主体直接相关。
\item[$\bullet$]第三,评价结果的正确与否依赖于对客体状况和主体需要的认识。
\item[$\bullet$]第四,价值评价有科学与非科学之别。  
\end{itemize}

\end{document}