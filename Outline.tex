\documentclass{ctexart}
\usepackage{listings}
\usepackage{fancyhdr}
\usepackage{geometry}
\usepackage{indentfirst}

\pagestyle{fancy}
\fancyhead[]{}
\cfoot{\thepage}
\renewcommand\headrulewidth{0pt}

\setlength{\parindent}{2em}
\geometry{a4paper,scale=0.8}    
\title{仰海锐老师班马原提纲}
\author{计算机科学与技术系 \ 朱卓然}
\date{\today}
\begin{document}
\maketitle

\section{名词解释}~{}
\subsection{马克思主义}~{}
马克思主义是由马克思和恩格斯创立并为后继者所不断发展的科学理论体系,
是关于自然、社会和人类思维发展一般规律的学说,是关于社会主义必然代替
资本主义、最终实现共产主义的学说,是关于无产阶级解放、全人类解放和每个人
自由而全面发展的学说,是无产阶级政党和社会主义国家的指导思想,是指引人民
创造美好生活的行动指南。

\subsection{交往}~{}
交往是唯物史观的重要范畴,指一定历史条件下的现实的个人、群体、阶级、民族、国家
之间在物质和精神上的相互往来、相互作用、彼此联系的活动。

\subsection{世界历史}~{}
唯物史观视域中的“世界历史”是指民族、国家通过普遍交往,打破孤立隔绝的
状态,进入相互依存、相互联系的世界整体化的历史。在马克思、恩格斯看来
,资本主义生产方式的发展和交往的普遍化推动了历史向世界历史的转变。

\subsection{金融资本}~{}
金融资本是由工业垄断资本和银行垄断资本融合在一起而形成的一种
垄断资本。
\subsection{金融寡头}~{}
金融寡头是指操纵国民经济命脉,并在实际上控制国家政权的少数垄断资本家或
垄断资本家集团。

\subsection{垄断利润}~{}
垄断利润是垄断资本家凭借其在社会生产和流通中的垄断地位而获得的超过平均利润的
高额利润。
\subsection{垄断价格}~{}
垄断价格包括垄断高价和垄断低价两种形式。垄断高价是指垄断组织出售商品时规定的高于生产价格的价格;
垄断低价是指垄断组织在购买非垄断企业所生产的原材料等生产资料时规定的低于生产价格的价格。
\vspace{+10mm}
\section{简答题}~{}
\subsection{感性认识和理性认识}~{}
\vspace{-5mm}
\subsubsection{感性认识和理性认识的概念}~{}
\vspace{-5mm}
\begin{itemize}
\item[$\bullet$]感性认识是认识的初级阶段,是人们在实践的基础上,通过感官直接感受到的关于事物的现象、事物的外部联系、事物的各个方面的认识。
\item[$\bullet$]理性认识是认识的高级阶段,是人们借助抽象思维,在概括整理大量的感性材料的基础上,达到关于事物的性质、全体、内部联系和事物自身规律性的认识。 
    
\end{itemize}

\subsubsection{感性认识和理性认识的辩证关系}~{}
\vspace{-5mm}
\\
相互区别:
\begin{itemize}
\item[$\bullet$]第一,对象不同。感性认识是对事物外部联系的反映,而理性认识是对事物内部联系的反映。
\item[$\bullet$]第二,形式不同。感性认识包括感觉、知觉和表象三种形式,而理性认识包括概念、判断和推理三种形式。
\item[$\bullet$]第三,特点不同。感性认识的特点是直接性和具体性,而理性认识的特点是间接性和抽象性。    
\end{itemize}
相互联系:
\begin{itemize}
    \item[$\bullet$]第一,理性认识依赖于感性认识。
    \item[$\bullet$]第二,感性认识有待于发展和深化为理性认识。
    \item[$\bullet$]第三,感性认识和理性认识相互渗透、相互包含。
    \end{itemize}

\subsubsection{从感性认识到理性认识的条件}~{}
\vspace{-5mm}
\begin{itemize}
\item[$\bullet$]第一,投身实践,深入调查,获取十分丰富和合乎实际的感性材料。这是实现由感性认识上升到理性认识的基础。
\item[$\bullet$]第二,经过思考的作用,运用理论思维和科学抽象,对丰富的感性材料进行去粗取精、去伪存真、由此及彼、由表及里的加工制作,形成概念和理论的系统。
\end{itemize}

\subsection{价值评价}~{}
\vspace{-5mm}

\subsubsection{价值评价的特点}~{}
\vspace{-5mm}
\begin{itemize}
\item[$\bullet$]第一,评价以主客体的价值关系为认识对象。
\item[$\bullet$]第二,评价结果与评价主体直接相关。
\item[$\bullet$]第三,评价结果的正确与否依赖于对客体状况和主体需要的认识。
\item[$\bullet$]第四,价值评价有科学与非科学之别。  
\end{itemize}

\subsubsection{价值评价的标准}~{}
对于任何主体而言,是否推动社会历史进步,是否符合社会发展趋势,
是否维护、满足了最广大人民的需要和根本利益,是价值评价的
最高标准,是判断特定主体实际需要是否合理的最高尺度。
\vspace{+10mm}
\subsection{价值规律}~{}
\vspace{-5mm}

\subsubsection{价值规律的主要内容和客观要求}~{}
商品的价值量由生产商品的社会必要劳动时间决定,商品交换
以价值量为基础,按照等价交换的原则进行。
\subsubsection{价值规律的作用}~{}
\vspace{-5mm}
\begin{itemize}
\item[$\bullet$]第一,自发地调节生产资料和劳动力在社会各生产部门之间的分配比例。
\item[$\bullet$]第二,自发地刺激社会生产力发展
\item[$\bullet$]第三,自发地调节社会收入的分配。
\end{itemize}
\subsubsection{价值规律的消极后果}~{}
\vspace{-5mm}
\begin{itemize}
\item[$\bullet$]第一,导致社会资源浪费。
\item[$\bullet$]第二,阻碍技术进步。
\item[$\bullet$]第三,导致收入两极分化。
\end{itemize}
\newpage

\subsection{垄断}~{}
\vspace{-5mm}

\subsubsection{垄断的内涵}~{}
所谓垄断,是指少数资本主义大企业,为了获得高额利润,过相互协议或联合,对一个或几个部门商品的生产、销售和价格进行操纵与控制。


\subsubsection{垄断产生的原因}~{}
\vspace{-5mm}
\begin{itemize}
\item[$\bullet$]第一,当生产集中发展到相当高的程度,极少数企业就会联合起来,操纵和控制本部门的生产和销售,实行垄断,以获得高额利润。
\item[$\bullet$]第二,激烈的竞争给竞争各方带来的损失越来越严重,为了避免两败俱伤,企业之间会达成妥协,联合起来实行垄断。
\item[$\bullet$]第三,生产高度集中,不但使原有中小企业无力与大企业竞争,而且在很大程度上限制新企业的进入,这也使少数大企业自然占据垄断地位。
\end{itemize}

\subsubsection{垄断使竞争更激烈的原因}~{}
\vspace{-5mm}
\begin{itemize}
\item[$\bullet$]第一,垄断没有消除产生竞争的经济条件。
\item[$\bullet$]第二,垄断必须通过竞争来维持。
\item[$\bullet$]第三,社会是复杂多样的,任何垄断组织都不可能把包罗万象的社会生产全部包下来。
\end{itemize}

\subsubsection{垄断条件下竞争的特点}~{}
\vspace{-5mm}
\begin{itemize}
\item[$\bullet$]在竞争目的上,垄断条件下的竞争是为获得高额垄断利润,并不断巩固和扩大自己的垄断地位和统治权利。
\item[$\bullet$]在竞争手段上,不仅采取经济手段还采取非经济手段,使竞争更加复杂、激烈。
\item[$\bullet$]在竞争范围上,不仅经济领域的竞争多种多样,而且还扩大到经济领域以外。
\end{itemize}

\newpage
\section{材料题}~{}
\subsection{社会革命的作用}~{}
\vspace{-5mm}
\begin{itemize}
\item[$\bullet$]首先,社会革命是实现社会形态更替的重要手段和决定性环节。
\item[$\bullet$]其次,社会革命能充分发挥人民群众创造历史的积极性和伟大作用。
\item[$\bullet$]而且,社会革命还能够极大地教育和锻炼包括革命阶级在内的广大人民群众。
\item[$\bullet$]最后,无产阶级革命将为消除阶级对抗,并充分利用全人的文明成果促进社会全面进步创造条件。 
\end{itemize}

\subsection{唯物辩证法和历史唯物主义}~{}
\vspace{-5mm}
\begin{itemize}
\item[$\bullet$]首先,社会革命是实现社会形态更替的重要手段和决定性环节。
\item[$\bullet$]其次,社会革命能充分发挥人民群众创造历史的积极性和伟大作用。
\item[$\bullet$]而且,社会革命还能够极大地教育和锻炼包括革命阶级在内的广大人民群众。
\item[$\bullet$]最后,无产阶级革命将为消除阶级对抗,并充分利用全人的文明成果促进社会全面进步创造条件。 
\end{itemize}

\subsection{经济全球化的问题}~{}
\vspace{-5mm}

\subsubsection{经济全球化的表现}~{}
\vspace{-5mm}
\begin{itemize}
\item[$\bullet$]生产全球化
\item[$\bullet$]贸易全球化
\item[$\bullet$]金融全球化
\end{itemize}

\subsubsection{经济全球化的动因}~{}
\vspace{-5mm}
\begin{itemize}
\item[$\bullet$]一是科学技术的进步和生产力的发展为经济全球化提供了坚实的物质基础和根本的推动力。
\item[$\bullet$]二是跨国公司的发展为经济全球化提供了适宜的企业组织形式。
\item[$\bullet$]三是各国的经济体制的变革和国际经济组织的发展是经济全球化的体制与组织保障。
\end{itemize}


\subsubsection{经济全球化的影响}~{}
\vspace{-5mm}
\\
积极影响:
\begin{itemize}
\item[$\bullet$]第一,为发展中国家提供先进技术和管理经验。
\item[$\bullet$]第二,为发展中国家提供更多的就业机会。
\item[$\bullet$]第三,推动发展中国家国际贸易发展。
\item[$\bullet$]第四,促进发展中国家跨国公司的发展。
\item[$\bullet$]经济全球化为世界经济增长提供了强劲动力,促进了商品和资本流动、科技和文明进步、各国人民交往。
\end{itemize}
消极影响:
\begin{itemize}
    \item[$\bullet$]第一,发达国家与发展中国家在经济全球化过程中的地位和收益不平等、不平衡
    \item[$\bullet$]第二,加剧了发展中国家在经济增长的同时出现资源短缺和环境污染。
    \item[$\bullet$]第三,一定程度上增加了经济风险。
    \end{itemize}

\end{document}