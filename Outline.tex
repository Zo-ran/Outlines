\documentclass{ctexart}
\usepackage{listings}
\usepackage{fancyhdr}
\usepackage{geometry}
\usepackage{indentfirst}

\pagestyle{fancy}
\lfoot{}
\renewcommand\headrulewidth{0pt}

\setlength{\parindent}{2em}
\geometry{a4paper,scale=0.8}
\title{仰海锐老师班马原提纲}
\author{计算机科学与技术系 \ 朱卓然}
\date{\today}
\begin{document}
\maketitle

\section{名词解释}~{}
\subsection{马克思主义}~{}
马克思主义是由马克思和恩格斯创立并为后继者所不断发展的科学理论体系,
是关于自然、社会和人类思维发展一般规律的学说,是关于社会主义必然代替
资本主义、最终实现共产主义的学说,是关于无产阶级解放、全人类解放和每个人
自由而全面发展的学说,是无产阶级政党和社会主义国家的指导思想,是指引人民
创造美好生活的行动指南。

\subsection{交往}~{}
交往是唯物史观的重要范畴,指一定历史条件下的现实的个人、群体、阶级、民族、国家
之间在物质和精神上的相互往来、相互作用、彼此联系的活动。

\subsection{世界历史}~{}
唯物史观视域中的“世界历史”是指民族、国家通过普遍交往,打破孤立隔绝的
状态,进入相互依存、相互联系的世界整体化的历史。在马克思、恩格斯看来
,资本主义生产方式的发展和交往的普遍化推动了历史向世界历史的转变。

\subsection{金融资本}~{}
金融资本是由工业垄断资本和银行垄断资本融合在一起而形成的一种
垄断资本。
\subsection{金融寡头}~{}
金融寡头是指操纵国民经济命脉,并在实际上控制国家政权的少数垄断资本家或
垄断资本家集团。

\subsection{垄断利润}~{}
垄断利润是垄断资本家凭借其在社会生产和流通中的垄断地位而获得的超过平均利润的
高额利润。
\subsection{垄断价格}~{}


\end{document}